\documentclass{article}


\begin{document}
\section{Bram: Title}

We are ... .... 
\section{Bram: Screen space fluid rendering}
Fluid is simulated using smoothed particle hydrodynamics, also called SPH.
This means that fluid is divided into a set of discrete elements.
SPH do not need a grid, this means that in theory there are no boundaries.
SPH is a lot faster than a traditional grid based techniques, a lot of physical concepts are much easier determined using SPH than using the traditional approach. 
An SPH simulation can be seen in this figure.

\section{Maarten:Curvature Flow}
Curvature describes the amount of smoothness of a surface.
Mean curvature flow can thus be used to smoothen a surface.
An example of mean curvature can be seen in this figure.

\section{...:Who, where and when?}
Method by ..., done at the RUG in collaboration with NVIDIA. 
It was developed in ....

\section{...:Why?}
It is a high quality simulation of fluid, which is used in games.
The main reason that this method was developed is that existing methods had several artifacts or computational issues.

\section{...:Related work}
An isosurface is created using marching cubes.
The disadvantages of using marching cubes are:
Works only in real-time with small amount of particles and is quite blobby.
A grid like structure is needed to perform marching cubes.


Metaball is a shape that is created using unions of spheres.
It is a implicit ray metaball algorithm.
Advantage of using metaballs is that it works without a grid like structure.
The disadvantages of using ray metaballs are:
Needs very big metaballs in order to decrease bumpyness.

\section{...:Overview of method}
A general overview of our method looks like this:
The input consists of positions of n particles which can be in any order.
...
...
A detailed explanation of each step is given in the upcoming slides.

\section{...:Surface depth}
We are interested in the surface, so we determine all the points that are closest to the viewpoint of the camera/

Refer to practical where normals and colors are splatted.
The reason for this is that the depth values will be manipulated by the smoothing step.
This means that the normals will also change in the smoothing step.
\end{document}