%!TEX root = ../report.tex

\section{Conclusion}
When the initial requirements for the realistic water rendering are considered it seems that the curvature flow way is a good fit.
\begin{itemize}
\item Given enough iterations of the fluid surface, a surface will become smooth and will have no sharp discontinuities between particles.
\item It is possible to generate an imperfect surface by adding a noise texture to the surface and attaching it to particles. This will distort the surface a little bit and will create an imperfect, and more natural surface
\item It takes into account attenuation of colors based on the depth of the fluids and submerged objects. Moreover, foam can be added for an added realistic effect.
\end{itemize}
Besides that it can produce a natural rendering of a fluid surface, it is also fast. 
It can render a smooth surface from thousands of particles and still have an acceptable framerate. 
This is also an important quality for a simulation.

\subsection{Results}

We have implemented the methods as described in the Implementation section.
The results are visible in figure~\ref{fig:res} and \ref{fig:res2}.

\begin{figure}[!th]
\hrule
\begin{center}
\vspace*{2ex}\includegraphics[width=0.48\textwidth,clip=true,trim=10cm 1cm 10cm 5cm]{pictures/colors_unsmoothed.png}
\end{center}
\caption{The SPH simulation with splatted point and Phong and Fresnel rendering without surface smoothing}
\label{fig:res} 
\vspace*{2ex}
\hrule
\end{figure}

\begin{figure}[!th]
\hrule
\begin{center}
\vspace*{2ex}\includegraphics[width=0.48\textwidth,clip=true,trim=10cm 1cm 10cm 5cm]{pictures/colors_smoothed.png}
\end{center}
\caption{The same simulation visualized with 24 smooth steps}
\label{fig:res2} 
\vspace*{2ex}
\hrule
\end{figure}
