%!TEX root = ../report.tex
\section{Implementation}
The solution that has been proposed in the previous section is implemented using the C++ programming language in combination with CG as shading language.
The actual SPH fluid simulation is given and can be seen in figure \ref{fig:sph}, by implementing the multiple passes described in the previous section, a nature like fluid is expected.

\subsection{Depth determination}
% TODO: Rephrase?
The depth at each pixel of each particle closest to the camera is determined using a combination of a vertex- and fragment shader.
Each particle has a position in world space that is passed to the vertex shader.
These particles are rendered as spheres to determine the correct depth values.

The vertex shader computes and passes the following properties of each particle to the fragment shader:
\begin{itemize}
 	\item Position of center in eye space.
 	\item Position of center in screen space.
 	\item Splat size.
 	\item Splat radius.
 \end{itemize} 

The fragment shader uses this input to determine the depth at every fragment.
To determine this depth, the splat is rendered as a sphere by discarding fragments that fall outside the sphere.
From this, the normal from the center of the sphere towards its surface is determined by taking the difference of the current fragment position and the current particle center.
Using this normal, the point is transformed to clip space, the $z$ value of this position is the depth value.

