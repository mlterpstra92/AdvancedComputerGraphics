%% BEGIN INCLUDE
\documentclass[a4paper, 10pt, twocolumn]{eguk2000}
\usepackage[english]{babel}
\usepackage[square]{natbib}
\usepackage{graphicx}
\usepackage{amsmath}
\usepackage{program}
\usepackage{algorithm}
\usepackage{algorithmicx}
\usepackage{algpseudocode}
\usepackage[hidelinks]{hyperref}


\begin{document}
%% END INCLUDE

\title{Screen Space Fluid Rendering with Curvature Flow}
\author{{\sffamily Maarten Terpstra${}^\ast$ and Bram Musters${}^\ast$}\\
\\
${}^\ast$Department of Computing Science\\
University of Groningen\\
Nijenborgh 9\\
Groningen 9747 AG\\
{\tt \{m.l.terpstra, b.t.musters\}@student.rug.nl}\\
\\
\parbox{140truemm}{\normalsize
{\bfseries Abstract}\\
Fluids can be simulated using Smoothed Particle Hydrodynamics(SPH). However, visualizing these fluids in a realistic way proves to be hard.
The paper by \cite{van2009screen} proposes a method that visualizes this SPH simulation as a nature-like fluid using surface smoothing by screen-space curvature flow.
The advantage of this method is that it can be rendered in real-time for a high amount of particles without a need for a finite grid.
This paper analyzes the proposed method and describes how to implement it using C++, OpenGL and the nVidia Cg toolkit.
}
\\
\\
\parbox{140truemm}{\normalsize
{\bfseries Keywords: Smoothed Particle Hydrodynamics, fluid, visualization, computer graphics}
}
} %end author

\date{}


\maketitle 

%!TEX root = ../report.tex
\section{Introduction}
% Sketch the context (what is it all about), what is the application domain (some pictures usually help here). Give required background knowledge.
Fluids are abundant in nature: whether close to our homes, falling from the sky, deep in the earth in the form of magma, or in large volumes as lakes or oceans, fluids are nearly guaranteed to be found everywhere. 
These fluids are not only necessary for most animals roaming the planet to survive, but are also used by humans in industrial processes to obtain power, starch, sawed wood or cooling in our engines.

For some applications it may be useful to simulate these fluids using computers in order to improve usage of the fluids in certain situations.
By using visualizations of fluids bottlenecks or missed opportunities in systems may be identified and the model can modified to improve the system.
This kind of visualizations are mainly used in engineering.

Other types of visualizations, which are mainly used in academia, includes fields of research as oceanography, volcanology and astrophysics. 
In these kind of visualizations, entire systems are modeled to identify breaking points or how different parameters may affect an entire system.

There are several methods available for simulating fluids, such as the Eulerian method or the Smoothed Particle Hydrodynamics (SPH). The Eulerian method is fairly straightforward, as it describes fluid motion as an integration of the surface over time. 
This simple technique comes with a downside in the fact that it requires a (finite) grid on which the fluid moves, which is computationally expensive and limits the fluid in that it cannot move everywhere.
SPH proves to be a viable alternative. 
With this technique, a body of fluid is reduced to a number of particles, where each particle has a position, velocity, acceleration and density.
These properties can be used to extract a smoothed surface which can then be visualized in an aesthetically pleasing way. 	

Simulating a fluid is only half the battle. The insight gained from a fluid rendering can increased by having a natural visualization of the system. 
%!TEX root = ../report.tex
\section{Problem Definition}
% Give a precise formulation of the problem you will be addressing
Since the fluids are simulated using SPH, the fluid is discretized in different particles.
A scene containing a fluid, simulated by SPH, can be seen in figure \ref{..}.
Since this ``fluid'' does not look like a fluid that can be found in nature, a visualization technique has to be applied.

\begin{figure}[!th]
\hrule
\begin{center}
\vspace*{2ex}\includegraphics[width=0.48\textwidth]{pictures/sph.png}
\end{center}
\caption{SPH simulation}
\label{fig:sph} 
\vspace*{2ex}
\hrule
\end{figure}

It is our goal to create a nature-like rendering of fluids that can be rendered in real-time, in order to use it in games, for example.
It is also desirable to make sure that the rendering can be customized according to the requirements of users.
For example, fluids can have different kind of thicknesses, or the quality of the rendering can be altered in order to keep the rendering in real-time.

\subsection{Related work}
Various methods are developed that try to achieve a nature-like rendering of fluid.
Some of these techniques require to use a mesh, which is not desirable.
Other techniques have the drawback that they can not be rendered in real-time.

\cite{zhang2008adaptive} developed a method that makes use of point-based rendering, therefore no grid is needed anymore.
However, a drawback of this method is that it results in unreasonably thick surfaces.

%!TEX root = ../report.tex
\section{Solution}
The paper introduced by van der Laan et al. \cite{van2009screen} provides a fitting solution to our problem.
The paper describes a method for visualizing fluids simulated using SPH in a natural way.
The goal of the paper is to provide a solution to our aforementioned problem.
According to the paper, they want to create:
\begin{enumerate}
	\item \label{en:item1} Achieves real-time performance, with a configurable speed versus quality trade-off.
	\item \label{en:item2} Does all the processing, rendering and shading steps directly on the graphics hardware.
	\item \label{en:item3} Smooths the surface to prevent the fluid from looking blobby or jelly-like.
	\item \label{en:item4} Is not based on polygonization, and thus does not suffer from the associated tessellation artifacts.
\end{enumerate}

Items \ref{en:item1} and \ref{en:item3} have a direct connection with our goals and items \ref{en:item2} and \ref{en:item4} are ways to enable these goals.

The method described in the paper consists of multiple passes. It assumes that there is a functional SPH simulation where each fluid particle has a density. Then each frame the following steps are performed:
\begin{enumerate}
	\item Splat points as spheres and determine depth values per fragment
	\item Smooth the spheres based on curvature flow
	\item Attenuate colors based on thickness
	\item Add noise texture and advect throughout the simulation
	\item Add foam
	\item Render using Fresnel and Phong equations
\end{enumerate}

\subsection{Depth determination}
It is desirable to determine depth of fragment in the simulation in order to know which fragments are part of the surface. 
In order to obtain these depth values, the points from the SPH simulation are splatted to discs and their fragments are subjected to a hardware depth test. 
This result is subsequently written to a depth buffer. 
Note that merely the depth value is splatted and the color and normal attributes are left unchanged as they will be changed in later steps.

\subsection{Smoothing}
Smoothing is a critical component in the paper. It is responsible for creating a smooth surface from a set of points so to avoid a blobby and jelly-like surface which is undesirable. The paper argues that this is a better approach than using Gaussian blurring because it performs better and ought to produce better results because there will be no silhouette blurring. 
This is achieved by translating points along the $z$-axis according the curvature flow. 

Curvature flow is defined by the divergence of the normal by \(2H = \nabla \cdot \hat{\mathbf{n}}\). Subsequently, depth values can be displaced every timestep based on the curvature in that point, as \(\frac{\partial z}{\partial t} = H\).

To obtain normals, the point of which the normal needs to be determined is mapped back to a point in view by inverting the projection transformation. This point is called % $\mathbf{P} = \Colvec[;]{a;b;c} $
$\Colvec{a}\Colvec{a,b}\Colvec[;]{a;b;c}$


\subsection{Thickness}

\subsection{Noise}

\subsection{Foam}

\subsection{Rendering}

\subsection{Fit}
%!TEX root = ../report.tex
\section{Implementation}
The solution that has been proposed in the previous section is implemented using the C++ programming language in combination with the OpenGL API.
To program on the GPU, CG is used as shading language.

The actual SPH fluid simulation is given and can be seen in figure \ref{fig:sph}, by implementing the multiple passes described in the previous section, a nature like fluid is expected.
By creating Frame Buffer Objects (FBO), the results can be written to an off-screen rendering target.
This is useful since we are using multiple passes.
Also, multiple buffers can be attached to the FBO.

\subsection{Depth determination}
% TODO: Rephrase?
The depth at each pixel of each particle closest to the camera is determined using a combination of a vertex- and fragment shader.
Each particle has a position in world space that is passed to the vertex shader.
These particles are rendered as spheres to determine the correct depth values.

The vertex shader computes and passes the following properties of each particle to the fragment shader:
\begin{itemize}
 	\item Position of center in eye space.
 	\item Position of center in screen space.
 	\item Splat size.
 	\item Splat radius.
 \end{itemize} 

The fragment shader uses this input to determine the depth at every fragment.
To determine this depth, the splat is rendered as a sphere by discarding fragments that fall outside the sphere.
From this, the normal from the center of the sphere towards its surface is determined by taking the difference of the current fragment position and the current particle center.
Using this normal, the point is transformed to clip space, the $z$ value of this position is the depth value.
The depth values are then written to the depth buffer of the current FBO.

The depth component of each fragment can be visualized as grey value by setting the R, G and B components to the depth value.
The results of this visualization can be seen in figure \ref{fig:depth}.
It can be observed that the particles become darker when they appear  closer to the camera.

\begin{figure}[!th]
\hrule
\begin{center}
\vspace*{2ex}\includegraphics[width=0.48\textwidth]{pictures/depth.png}
\end{center}
\caption{Depth component visualization}
\label{fig:depth} 
\vspace*{2ex}
\hrule
\end{figure}

\subsection{Surface smoothing}
Surface smoothing is implemented using a fragment shader.
Since the smoothing happens in multiple steps, two depth textures are needed.
The shader smooths the depth values of the main depth texture and outputs them in the second alternate depth texture.
Then the textures are flipped so the main depth texture has newly calculated values, this process repeats until the desired amount of smoothing steps is reached.

% TODO: Fix dit
First, the normal at each fragment has to be determined.
Using the normals, the curvature has to be calculated, and this needs to be smoothed.

\subsubsection{Normal calculation}
The depth buffer is passed as input to the fragment shader, along with the uniform variables $C_x$ and $C_y$.
Since the depth buffer is in a texture, we can obtain the depth values of neighbours.
Using these neighbours, the finite differences between depth values can be calculated.
Since we now have $C_x, C_y, \frac{\partial z}{\partial x}, \frac{\partial z}{\partial y}$ and $z$, we can calculate the normal according to equation \ref{eq:normals}.

\subsection{Normal smoothing}

\subsection{Rendering}
For the final rendering phase we implemented a fragment shader that produces a color for each fragment, which only has a depth component.
We have a default color for water that is modified according to its surroundings based on shading. 
The end-results are projected onto a full-screen quad, which is the best option due to our framebuffer-objects.

The paper suggests to implement a combination of Phong shading and the Fresnel equations shading for displaying the surface. 

Phong shading works by linearly interpolating the normals from the vertices across a surface\cite{phong1975illumination}. 
This results in a smooth surface of an object with correct specular highlights, as opposed to Gouraud shading which is not smooth and may not render specular highlights.

The Fresnel equations describe the the behaviour of light when it moves between media with different refractive indices. 
This helps visualizing water in a realistic way because the rendering can take refraction and reflection of a fluid surface into account in a realistic way.

Normally, computing the Fresnel equations would be a costly and difficult operation. To circumvent this, we have applied Schlick's approximation to determine the reflective Fresnel component.

%!TEX root = ../report.tex
\section{Results}
We have implemented the methods as described in the Implementation section.
Our project is not quite done, but the first results are visible in figure~\ref{fig:res}


\begin{figure}[!th]
\hrule
\begin{center}
\vspace*{2ex}\includegraphics[width=0.48\textwidth]{pictures/colors.png}
\end{center}
\caption{Shaded results}
\label{fig:res} 
\vspace*{2ex}
\hrule
\end{figure}
\section{Workload division}
All the programming was done in pairs so we'd value that the programming was 50-50 effort.

For the report, Bram focused more on problem definition, and implementation while Maarten focused more on the solution as proposed by the paper and the introduction.

All in all, we feel that we both contributed equally to the entire final project.


%% BEGIN INCLUDE
\bibliographystyle{eguk2000}
\bibliography{references}
\end{document}
%% END INCLUDE
